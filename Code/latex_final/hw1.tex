%\input{preamble_CJK}   % 使用自己維護的定義檔
%%-----------------------------------------------------------------------------------------------------------------------
%% 文章開始
%\title{\LaTeX{\MB 的數學符號與方程式}}
%\author{{\SM 陳威傑}}
%\date{{\TT \today}} 	% Activate to display a given date or no date (if empty),
%         						% otherwise the current date is printed 
%\begin{document}
%\maketitle
\fontsize{12}{22 pt}\selectfont
\chapter{數學函式}
\LaTeX 被人津津樂道的就是他對於數學函數的編排方式,雖然看起來一堆程式,也有一堆原則必須遵守,但最後編排出來的美感就是他最獨一無二的地方,也因為是經由編寫程式的方式進行,就像很多程式語言一樣,最一開始總是最痛苦的,但只要跟著本章節的編排順序一步一步慢慢來,就可以由淺入深的進入\LaTeX 的世界中。
\section{{輸入數學函式}}
輸入數學函式的方式大致可以分為以下兩種:
\begin{enumerate}
\item 在文字中間\\
通常在文字中使用兩個 "\$"符號將數學式包含其中,像是若我輸入 \$2x+5\$即會出現 $2x+5$
\item 獨立一行\\
若需要讓函數獨立一行,則會在函數前後使用 "\$\$"將函數包圍起來,若我輸入\$\$2x+5\$\$,則畫面便會出現$$2x+5$$\\亦可以使用下列方法達到讓函數獨立一行的目的
\begin{center}
$\backslash$begin$\{$equation$\}$ ... $\backslash$end$\{$equation$\}$\\
$\backslash$begin$\{$displaymath$\}$ ... $\backslash$end$\{$displaymath$\}$\\
$\backslash$[ ... $\backslash$]\\
        \end{center}
\end{enumerate}
\section{{數學函式對齊}}
輸入函式之後,\LaTeX 也有內建的指令可以對函數的對齊方式進行調整
\begin{enumerate}
\item 使用align指令\\
使用指令$\{$align$\}$來進行函數對齊時,在要對齊的地方加入\& 即可讓函數對齊,例如我輸入:
$\backslash$ begin $\{$align$\}$ a \&=b+c $\backslash$ $\backslash$d\& =e+f $\backslash$ end $\{$ align $\}$\\
便會將兩函數的"="進行對齊\begin{align} a &=b+c\notag\\ d &=e+f+g+h\notag 
\end{align}
\item 使用eqnarray指令\\
使用指令$\{$eqnarray$\}$來進行函數對齊時,在要對齊的地方前後各加入"\&" 即可讓函數對齊,例如我輸入:\\
$\backslash$begin $\{$eqnarray$\}$a \&=\& b+c$\backslash\backslash$d=e+f+g+h+i+j$\backslash$end$\{$eqnarray$\}$\\
便會將兩函數的"="進行對齊\begin{eqnarray}a&=&b+c\nonumber\\d&=&e+f+g+h+i+j\nonumber
\end{eqnarray}
\item 使用gather指令\\
使用指令$\{$gather$\}$可讓兩組數學函數居中對齊,例如我輸入:\\
$\backslash$ begin $\{$gather$\}$ a=b+c$\backslash\backslash$d=e+f$\backslash$ end $\{$gather$\}$ $\backslash$\\
便會在下一行出現\begin{gather}
a=b+c\notag \\ d=e+f+i+j\notag 
\end{gather}
\end{enumerate}

\section{{數學式中的文字}}
\LaTeX 在進行文字與數學符號交錯邊排的時候,需要特別注意文字的表現,通常在數學函式中的文字會以斜體呈現,此時,僅需在函數結尾的"\$"或"\$\$"符號之前輸入即可,如:\$\$x+5\;, where  x $\backslash$leq 2\$\$,即會顯現$$x+5, where \; x \leq 2$$\\
但若不要讓文字產生斜體效果,則需要使用$\backslash$mbox$\{$...$\}$的方式,在$\{...\}$中輸入文字,如\$\$x+5\;$\backslash$mbox$\{$, where$\}$  x $\backslash$leq 2\$\$,即會出現$$x+5 \mbox{, where} \; x \leq 2$$\\

\section{{符號使用}}
符號的使用在\LaTeX 中可以說是一個非常重要的學問,因為這些符號的使用在\LaTeX 多半有自己的定位,而不是將該符號程現在文字編排中,也有些特殊的數學符號需要特殊的輸入才能呈現
\subsection{括號}
\begin{enumerate}
\item \textcolor{bole}{小括號}\\
在呈現小括號時,不能僅打 $(...)$符號,而是需要像數學函數一樣,輸入\$$(...)$\$才可以顯示出來\\
\item \textcolor{bole}{中括號}\\
在呈現中括號時,不能僅打 $[...]$符號,而是需要像數學函數一樣,輸入\$$[...]$\$才可以顯示出來\\
\item \textcolor{bole}{大括號}\\
在呈現大括號時,不能僅打 $\{...\}$符號,而是需要像數學函數一樣,輸入\$${...}$\$才可以顯示出來\\
\item \textcolor{bole}{上括號}\\
在呈現上括號時,需要輸入$\backslash$overbrace$\{...\}$才可以在字串上顯示上括號\\
\item \textcolor{bole}{下括號}\\
在呈現下括號時,需要輸入$\backslash$underbrace$\{...\}$才可以在字串下顯示括號\\
\end{enumerate}
\subsection{分數與組合符號}
\subsubsection{分數}
\begin{enumerate}
\item 分數的函數公式:\\
\begin{center}
$\backslash$ frac $\{$ 分子$\}$ $\{$ 分母$\}$
\end{center}
\item 範例:\\
我輸入\$\$$\backslash$ frac $\{$ 2$\}$ $\{$ 7$\}$\$\$,便會在下一行出現$$\frac{2}{7}$$
\item 練習:\\
輸入 $\beta$ $(a, b)$
\begin{equation} \label{frac2}
\frac{(a-1)!(b-1)!}{(a+b-1)!}
\end{equation}

\end{enumerate}

\subsubsection{組合符號}
\begin{enumerate}
\item 組合符號的函數公式:\\
\begin{center}
$\{$ $($總數$)$ $\backslash$ choose $($取的個數$)$ $\}$
\end{center}
\item 範例:\\
我輸入\$\$$\{$ $($n$)$ $\backslash$ choose $($x$)$ $\}$\$\$,便會在下一行出現$$ {n\choose x}$$
\item 練習:\\
輸入超幾何分配 $f(k;n,K,N)$
\begin{center}\label{frac3}
$$\frac{{K\choose k}{N-K\choose n-k}}{{N\choose n}}$$
\end{center}
\end{enumerate}
\subsection{積分與極限}
\subsubsection{積分}
\begin{enumerate}
\item 積分符號的函數公式:\\
\begin{center}
$\backslash$ int\textunderscore  $\{$ 積分下界$\}$  \textasciicircum $\{$ 積分上界$\}$
\end{center}
\item 範例:\\
我輸入\$\$$\backslash$ int\textunderscore  $\{$ -2$\}$  \textasciicircum $\{$ 1$\}$ f(x)dx\$\$,便會在下一行出現$$\int_{-2}^{1}f(x)dx$$
\item 練習:\\
輸入累積常態機率密度函數 $F(x), x\sim N(\mu, \sigma^2)$
\begin{equation}\label{int1}
\int_{-\infty}^{x}\frac{1}{\sqrt{2\pi}\sigma}e^\frac{(t-\mu)^2}{2\sigma^2}dt
\end{equation}

\end{enumerate}
\subsubsection{極限}
\begin{enumerate}
\item 極限符號的函數公式:\\
\begin{center}
$\backslash$lim\textunderscore$\{$取極限的參數$\backslash$rightarrow 參數極限逼近值$\}$ $\backslash$\;(方程式)
\end{center}
\item 範例:\\
我輸入\$\$$\backslash$lim\textunderscore$\{$x$\backslash$rightarrow 2$\}$ $\backslash$\;(x+2)\$\$,便會在下一行出現$$\lim_{x\rightarrow 2} x+2$$
\item 練習:\\
輸入機率收歛公式 $X_{n}\stackrel{p}{\longrightarrow} X $
\begin{equation}\label{int2}
\lim_{n\rightarrow \infty} (\mathbb{P}(|X-X_{n}|\geq\varepsilon)=0) 
\end{equation}
\end{enumerate}
\subsection{累加與累乘}
\subsubsection{累加}
\begin{enumerate}
\item 累加符號的函數公式:\\
\begin{center}
$\backslash$ sum\textunderscore  $\{$ 累加下界$\}$  \textasciicircum $\{$ 累加上界$\}$
\end{center}
\item 範例:\\
我輸入\$\$$\backslash$ sum\textunderscore  $\{$x=2$\}$  \textasciicircum $\{$5$\}$2x+1\$\$,便會在下一行出現$$\sum_  {x=2}^{5}2x+1$$
\item 練習:\\
輸入累積間斷函數期望值公式E(x), $x=1,2,\cdots n$
\begin{equation}\label{sum1}
\sum_{x=1}^{n}xf(x)
\end{equation}
\end{enumerate}
\newpage
\subsubsection{累乘} 
\begin{enumerate}
\item 累乘符號的函數公式:\\
\begin{center}
$\backslash$ prod\textunderscore  $\{$累乘下界$\}$  \textasciicircum $\{$累乘上界$\}$
\end{center}
\item 範例:\\
我輸入\$\$$\backslash$prod\textunderscore  $\{$i=2$\}$  \textasciicircum $\{$5$\}$ 2x \textunderscore i+1\$\$,便會在下一行出現$$\prod_{i=2}^{5} 2x_{i}+1$$
\item 練習:\\
輸入最大概似含數
\begin{equation}\label{sum2}
\prod_{i=1}^{n} f(x_i)
\end{equation}
\end{enumerate}

\section{矩陣}
\begin{enumerate}
\item 矩陣符號的函數公式:\\
\begin{center}
\textbf{$\backslash$begin$\{$array$\}$ $\{$...$\}$  a\&b $\backslash$end$\{$array$\}$}
\end{center}
\underline{$\{$...$\}$中放入與所需欄位相同的英文字母數量,接續每一欄的資料輸入完畢之後,}\\
須使用"\&" 符號進行換欄的動作
\item 範例:\\
我輸入\emph{\$\$$\backslash$begin$\{$array$\}$ $\{$lr$\}$  a\&b$\backslash\backslash$c\&d$\backslash$end$\{$array$\}$\$\$},便會在下一行出現$$\begin{array}{lr} 
 a&b\\c&d
 \end{array}$$
 \end{enumerate}
\subsection{矩陣括號}
\begin{itemize}
\item 大型小括號輸入法
\begin{center}
\emph{\textbf{$\backslash$lfet$($....$\backslash$ right $)$}}
\end{center}
\item 大型中括號輸入法
\begin{center}
\emph{\textbf{$\backslash$lfet$[$.... $\backslash$right $]$}}
\end{center}
\item 大型大括號輸入法
\begin{center}
\emph{\textbf{$\backslash$lfet$\{$.... $\backslash$right $\}$}}
\end{center}
\item 練習:\\
輸入大型中括號的2*2單位矩陣
        $$ I_{2*2} = \left[
            \begin{array}{lr}
                1 & 0\\
                0 & 1\\
            \end{array} \right] $$

\end{itemize}
\subsection{删節點} 
\begin{enumerate}
\item \textcolor{green!10!blue!90!}{橫向刪節點:}\\
\colorbox{orange}{使用$\backslash$cdots便可以輸入橫向刪節點}
\item \textcolor{green!10!red!90!}{縱向刪節點:}\\
\colorbox{pink}{使用$\backslash$vdots便可以輸入橫向刪節點}
\item \textcolor{green!10!orange!90!}{斜向刪節點:}\\
\colorbox{alizarin}{使用$\backslash$ddots便可以輸入橫向刪節點}
\end{enumerate}
\newpage
\section{{矩陣與數學函數綜合練習}}
\begin{try}\;\;Prove $E(x)=\lambda$ and $Var(x)=\lambda$ for Poission distrubution with parameter=$\lambda$, check equation below
\begin{align}
 E(x) &= \sum_{k=1}^{\infty}k f(k)\notag= \sum_{k=1}^{\infty}k\frac{e^{-\lambda}\lambda^{k}}{k!}= \sum_{k=1}^{\infty}\frac{e^{-\lambda}\lambda^{k}}{x_k!}\notag\\[6mm]
      &= \sum_{k=1}^{\infty}\frac{\lambda e^{-\lambda}\lambda^{k-1}}{(k-1)!}=
      \lambda \sum_{k=1}^{\infty}\frac{e^{-\lambda}\lambda^{k-1}}{(k-1)!}=
      \lambda\underbrace{\sum_{k=1}^{\infty}\frac{e^{-\lambda}\lambda^{k-1}}{(k-1)!}}_{\text{=$\sum_{j=k-1}^{\infty}\frac{e^{-\lambda}\lambda^{j}}{j!}=1$}}\notag\\
      &= \lambda\notag\\
       E(x^2) &= \sum_{k=1}^{\infty}k^2 f(k)\notag= \sum_{k=1}^{\infty}k^2\frac{e^{-\lambda}\lambda^{k}}{k!}= \lambda\sum_{k=1}^{\infty}k\frac{e^{-\lambda}\lambda^{k-1}}{(k-1)!}\\[6mm]
     &= \lambda e^{-\lambda}\left(\sum_{k=1}^{\infty}(k-1)\frac{1}{(k-1)!}\lambda^{k-1}+\sum_{k=1}^{\infty}\frac{1}{(k-1)!}\lambda^{k-1}\right)\notag\\
&= \lambda e^{-\lambda}\left(\lambda\sum_{k=2}^{\infty}\frac{1}{(k-2)!}\lambda^{k-2}+\sum_{k=1}^{\infty}\frac{1}{(k-1)!}\lambda^{k-1}\notag\right)\\
&= \lambda e^{-\lambda}\left(\lambda\sum_{i=k-2}^{\infty}\frac{1}{i!}\lambda^i+\sum_{j=k-1}^{\infty}\frac{1}{j!}\lambda^j\notag\right)\\
&=\lambda e^{-\lambda}(\lambda e^{\lambda}+e^{\lambda})=\lambda(\lambda+1)=\lambda^2+\lambda\notag\\
Var(x)&=E(x^2)-(E(x))^2=\lambda^2+\lambda-(\lambda)^2=\lambda\notag
\end{align}
\end{try}

\newpage
 \begin{try}二維常態下,證明$\rho_{XY}$=$\rho$, 參考證明$(\ref{rho})$
\begin{align}\label{rho}
x \sim N(&\mu_{x}, \sigma_{x}^2)\;\;\;\;y \sim N(\mu_{y}, \sigma_{y}^2),\; x\;\mbox{and}\;y \;\mbox{are indepandent}\notag\\[4mm]
f(x, y)&=(2\pi\rho_{x}\rho_{y}\sqrt{1-\rho^2})^{-1}\notag\\\notag[5mm]
&*\mbox{exp}\left(-\frac{1}{2(1-\rho^2)}\left(
(\frac{x-\mu_{x}}{\sigma_{x}})^2-2\rho(\frac{x-\mu_{x}}{\sigma_{x}})(\frac{y-\mu_{y}}{\sigma_{y}})+(\frac{y-\mu_{y}}{\sigma_{y}}^2)\right)\right)\\\notag
\mbox{And} & \\\notag
\rho_{xy} &= \frac{\mbox{Cov}(x, y)}{\sigma_{x}\sigma_{y}}=\frac{\mbox{E}(x-\mu_{x})(y-\mu_{y})}{\sigma_{x}\sigma_{y}}=\mbox{E}(\frac{x-\mu_{x}}{\sigma_{x}})(\frac{y-\mu_{y}}{\sigma_{y}})\\\notag[6mm]
&=\int_{-\infty}^{\infty}\int_{-\infty}^{\infty}(\frac{x-\mu_{x}}{\sigma_{x}})(\frac{y-\mu_{y}}{\sigma_{y}})f(x, y)\;dxdy\\\notag[6mm]
\mbox{Let} \;\;s &=(\frac{x-\mu_{x}}{\sigma_{x}})(\frac{y-\mu_{y}}{\sigma_{y}})\;\;\mbox{and} \;\; t=(\frac{x-\mu_{x}}{\sigma_{x}})\\\notag
\mbox{Then} \; x &=\sigma_{x}t, \; y=(\sigma_{y}s/t), \; \mbox{and Jacobian}\; J=\sigma_{x}\sigma_{y}/t,\; \mbox{rewrite function}\\\notag[6mm]
\rho_{xy}&=\int_{-\infty}^{\infty}\int_{-\infty}^{\infty}sf\left(\sigma_{x}t+\mu_{x},\;\frac{\sigma_{y}s}{t}+\mu_{y}\right)\vert\frac{\sigma_{x}\sigma_{y}}{t}\vert \;dsdt\\\notag[6mm]
&=\int_{-\infty}^{\infty}\int_{-\infty}^{\infty}s\left(2\pi\sigma_{x}\sigma_{y}\sqrt{1-\rho^2}\right)^{-1}\\
&*\mbox{exp}\left(-\frac{1}{2(1-\rho^2)}(t^2-2\rho s+(\frac{s}{t})^2\right)\frac{\sigma_{x}\sigma_{y}}{\vert t\vert} \;dsdt\\\notag[6mm]
\mbox{Because}\;\vert t\vert &=\sqrt{t^2}\;\mbox{and}\; t^2-2\rho s+(\frac{s}{t})^2=(\frac{s-\rho t^2}{t})^2+(1-\rho^2)t^2,\; \mbox{rewrite function}\\\notag[6mm]
\rho_{xy} &= \int_{-\infty}^{\infty}\frac{1}{\sqrt{2\pi}}\mbox{exp}(-\frac{t^2}{2})\left[\int_{-\infty}^{\infty}\frac{s}{\sqrt{2\pi}\sqrt{(1-\rho^2)t^2}}\mbox{exp}\left(-\frac{(s-\rho t^2)^2}{2(1-\rho^2)t^2}\right)ds\right]dt\\\notag[6mm]
s &\sim N(\rho t^2, (1-\rho^2)t^2),\;\int_{-\infty}^{\infty}\frac{s}{\sqrt{2\pi}\sqrt{(1-\rho^2)t^2}}\mbox{exp}\left(-\frac{(s-\rho t^2)^2}{2(1-\rho^2)t^2}\right)ds=1\\\notag[4mm]
\mbox{Then}\; \rho_{xy} &=\int_{-\infty}^{\infty}\frac{\rho t^2}{\sqrt{2\pi}}\mbox{exp}(-\frac{t^2}{2})\; dt=\rho \underbrace{\int_{-\infty}^{\infty}\frac{t^2}{\sqrt{2\pi}}\mbox{exp}(-\frac{t^2}{2})\; dt}_{\int_{-\infty}^{\infty}N(0,1)\;dt=1}\\\notag[2mm]
\mbox{Prove} \;&\;\rho_{xy}=\rho*1=\rho\\\notag
\end{align} 
\end{try}\newpage
\begin{try} Use MGF of Normal distrubution to prove $E(x)$=$\mu$ and $Var(x)$=$\sigma^2$, check equation$(\ref{normal})$
\begin{align}\label{normal}
x \sim N&(\mu_{x}, \sigma_{x}^2)\notag\\
 M_x(t)&=E(e^{tx}) = \int e^{tx}\frac{1}{\sqrt{2\pi\sigma^2}}e^{-\frac{(x-\mu)^2}{2\sigma^2}}dx\notag\\[4mm]
 \mbox{Let}\; z&=\frac{x-\mu}{\sigma}\;\; \mbox{, which implies}\;\; x = z\sigma + \mu\notag\\[4mm]
 M_x(t)&=e^{\mu t}\int e^{zt\sigma}\frac{1}{\sqrt{2\pi\sigma^2}}e^{-\frac{1}{2}z^2}\mid\frac{dx}{dz}\mid dz\notag\\[4mm]
 &=e^{\mu t}\int e^{zt\sigma}\frac{1}{\sqrt{2\pi}}e^{-\frac{1}{2}z^2}dz\notag\\[4mm]
&=e^{\mu t} e^{\frac{1}{2}\sigma^2t^2}\notag \;\;=e^{\frac{1}{2}\sigma^2t^2+\mu t}\notag\\[4mm]
  M_x'(t)&=\frac{dx}{dt}M_x(t)=e^{\frac{1}{2}\sigma^2t^2+\mu t}dx=(\frac{1}{2}\sigma^2(2t)+\mu)e^{\frac{1}{2}\sigma^2t^2+\mu t}\\[4mm]
  &=(\sigma^2t+\mu)e^{\frac{1}{2}\sigma^2t^2+\mu t}\notag\\[4mm]
  E(x)&=M_x'(0)=(0+\mu)e^0=\mu\notag\\[4mm]
    M_x''(t)&=\frac{dx}{dt}M_x'(t)=(\sigma^2t+\mu)e^{\frac{1}{2}\sigma^2t^2+\mu t}\notag\\[4mm]
    &=(\sigma^2)e^{\frac{1}{2}\sigma^2t^2+\mu t}+(\sigma^2t+\mu)(\frac{1}{2}\sigma^2(2t)+\mu)e^{\frac{1}{2}\sigma^2t^2+\mu t}\notag\\[4mm]
    &=(\sigma^2)e^{\frac{1}{2}\sigma^2t^2+\mu t}+(\sigma^2t+\mu)(\sigma^2 t+\mu)e^{\frac{1}{2}\sigma^2t^2+\mu t}\notag\\[4mm]
  Var(x)&=M_x''(0)-E(x)^2=(\sigma^2+\mu^2)-\mu^2=\sigma^2\notag\\[4mm]
  \mbox{Prove}\; E&(x)=\mu \;\mbox{and}\;Var(x)\sigma^2\notag\\\notag
\end{align}
\end{try} 
\section*{小結}
在數學函數的輸入過程中,其實可以對照微軟的Microsoft系列,會發現雖然使用\LaTeX 真的比較困難一點,尤其常常括號過多的時候,少一個就會跑錯誤訊息,但是編排出來的文件真的有數學式的美感,學習的過程雖然辛苦,但是也會得到很好的結果。
\chapter{標號與參照}
在進行數學類文件的編排時,常常會遇到一堆不同的數學函式,此時就需要借助「標號」的功能,讓讀者能更快速且便捷的搜尋,同時,在文章內容引用時,也有自動化的「參照」功能讓標號自動的輸入在文件中。
\section{標號}
\subsection{輸入標號}
在對方程式進行標號的過程中,需要用到$\backslash$begin$\{$equation$\}$ ... $\backslash$end$\{$equation$\}$或$\backslash$begin$\{$displaymath$\}$ ... $\backslash$end$\{$displaymath$\}$的輸入方式,接著在$\backslash$begin$\{$equation$\}$或$\backslash$begin$\{$displaymath$\}$後方加入$\backslash$label$\{$...$\}$,在...中輸入函數名稱,這樣便可以自動的更新函數編號,例如當我輸入$\backslash$begin$\{$equation$\}$ $\backslash$label$\{$ex1$\}$ 2x+1 $\backslash$end$\{$equation$\}$便會出現\begin{equation}\label{ex1}
2x+1
\end{equation}
\subsection{調整標號位置}
\begin{enumerate}
\color{coolblack}
\item 使用align指令\\
\colorbox{bananamania}{\textcolor{coolblack}{使用套件 $\{$amsmath$\}$中指令$\{$align$\}$的$\backslash$notag,即可讓該行函數無標號,}}
例如我輸入$\backslash$ begin $\{$ align$\}$ a=b+c $\backslash$ notag $\backslash$ $\backslash$d=e+f $\backslash$ end $\{$ align $\}$,便會在下一行出現\begin{align} a=b+c\notag\\ d=e+f 
\end{align}
\item 使用eqnarray指令\\
使用指令$\{$eqnarray$\}$的$\backslash$nonumber,即可讓該行函數無標號,
例如我輸入$\backslash$ begin $\{$eqnarray$\}$ a=b+c$\backslash$ $\backslash$d=e+f $\backslash$nonumber $\backslash$end $\{$eqnarray$\}$,便會在下一行出現\begin{eqnarray} a=b+c\\ d=e+f\nonumber
\end{eqnarray}
\item 使用split指令\\
使用指令$\{$equation$\}$與$\{$split$\}$指令,可讓標號出現在兩數學函數中,\\
例如我輸入$\backslash$ begin $\{$equation$\}$ $\backslash$ begin $\{$split$\}$ a=b+c$\backslash$ $\backslash$d=e+f$\backslash$ end $\{$split$\}$end $\{$equation$\}$ $\backslash$ ,便會在下一行出現\begin{equation}
\begin{split}
a=b+c\\ d=e+f
\end{split}
\end{equation}
\end{enumerate}
\section{參照}
在使用完函數的標號之後,,若我們需要在文件中進行參照,僅須在文件中輸入$\{$ $\backslash$ref 標號時的函數名稱$\}$,即會在文件中自動對該方程式進行參照與調整,例如我要引用上方標號的範例公式ex1,我便可以輸入$\{$ $\backslash$ref ex1$\}$,即可出現該函數的標號{\ref{ex1}},可另外自行輸入括號。\\
\section*{小結}
標號與參照的學習過程雖然麻煩,但其實也間接在培養我們命名的能力,尤其在程式語言中,我們若能精準的對元件進行命名,不僅可以縮短我們尋找的時間,當分享給別人的時候,對方也能較迅速的知道我們對每個元件定義為何,從假設abc到一個較有意義的名稱可以說是一個壞習慣更改的過程,但也是一定要走的路。


%\end{document}
