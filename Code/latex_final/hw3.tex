%input{preamble}   % �ϥΦۤv���@���w�q��
%-----------------------------------------------------------------------------------------------------------------------
%% �峹�}�l
%\title{ \LaTeX  {\MB 簡報}}
%\author{{\SM 陳威傑}}
%\date{{\TT \today}} 	% Activate to display a given date or no date (if empty),
%         						% otherwise the current date is printed 
%\begin{document}
%\maketitle
%\fontsize{12}{22 pt}\selectfont
\chapter{簡報}
 在許多統計課堂或是研討會上,我們常常會看到使用 \LaTeX 做的簡報,這樣的簡報形式,雖然較使用Power Point來的樸素,但是由於他在數學函數上較美觀的編譯模式,因此仍在學術界佔有一定的地位,以下我們以臺北大學汪群超老師的教學網頁\footnote{汪群超老師\LaTeX 教學官方網站 https://ntpuccw.blog/supplements/xetex-tutorial}做為主要的範例。
\section{簡報套件}
\LaTeX 簡報的套件主要的套件是{\A beamer}套件,並同時需要在preamble中設定$\backslash$documentclass$\{$beamer$\}$
\section{簡報樣式}
在套件{\A beamer}中提供許多設定好的簡報樣式,在preamble中設定$\backslash$usetheme$\{$樣式名稱$\}$即可,同時,也可以使用$\backslash$usecolortheme$\{$色彩配置名稱$\}$調整樣式裡的色彩配置狀態,以下列舉一些樣式名稱與其呈現效果,更多的樣式與色彩配置可以參考網路資源\footnote{https://hartwork.org/beamer-theme-matrix/}\\
\begin{figure}[H]
  \centering
  \subfloat[usetheme=Berlin]{\includegraphics[scale=0.25]{\imgdir{Berlin.png}}}
  \subfloat[usetheme=Warsaw]{\includegraphics[scale=0.25]{\imgdir{Warsaw.jpg}}}\\
  \subfloat[usetheme=Darmstadt]{\includegraphics[scale=0.25]{\imgdir{Darmstadt.png}}}
  \subfloat[usetheme=Boadilla]{\includegraphics[scale=0.25]{\imgdir{Boadilla.png}}}\caption{不同usetheme}
\end{figure}
\begin{figure}[H]
  \centering
  \subfloat[usecolortheme=wolverine]{\includegraphics[scale=0.25]{\imgdir{wolverine.png}}}
  \subfloat[usecolortheme=dove]{\includegraphics[scale=0.25]{\imgdir{dove.png}}}\\
  \subfloat[usecolortheme=dolphin]{\includegraphics[scale=0.25]{\imgdir{dolphin.png}}}
  \subfloat[usecolortheme=fly]{\includegraphics[scale=0.25]{\imgdir{fly.png}}}\caption{不同usecolortheme}
\end{figure} 
\section{首頁}
簡報的首頁通常會包含簡報標題、作者、單位與時間,,而在此同時,也同時會設定個頁頁首或是頁尾的呈現形式,可參考圖\ref{fig:title}\\
\begin{table}[H]\caption{首頁呈現}\label{wide}
    \centering
\begin{tabular}{cp{8cm}}
\extrarowheight=10pt
\\\hline
程式 & 效果\\
\hline
  $\backslash$ title $[$ A $]$ $\{$ $\{$B$\}$ $\}$  & 標題,會將A呈現在簡報的最下方中間的位置,將B呈現在簡報起始頁
\\\hline  % &代表換欄 \\代表換下一列
  $\backslash$ subtitle $[$ A $]$ $\{$ $\{$B$\}$ $\}$  & 副標題,會將A呈現在簡報起始頁做為副標題
\\hline
  $\backslash$ author $[$ A $]$ $\{$ $\{$B$\}$ $\}$  & 作者,會將A呈現在簡報的最下方中間的位置,將B呈現在簡報起始頁
\\\hline
  $\backslash$ institute $[$ A $]$ $\{$ $\{$B$\}$ $\}$  & 單位,會將A呈現在簡報的最下方中間的位置,將B呈現在簡報起始頁
\\\hline
  $\backslash$date$\{$日期$\}$ & 顯示日期,若為製作當天,可使用$\backslash$today,也可以使用$\backslash$empty讓日期不顯示  \\\hline
    \end{tabular}
    \end{table}
    \begin{figure}[H]
    \centering
    \includegraphics[scale=0.7]{\imgdir{title.jpg}}
    \caption{首頁設定}\label{fig:title}
    \end{figure}
\section{簡報內容}
進行簡報內容輸入前,需使用$\backslash$frame$\{$簡報內容$\}$,另外,因目錄的製作與\LaTeX 中{\A section}相關的函數有關,因此,若需要自動創造簡報目錄,需在$\backslash$frame$\{$簡報內容$\}$前加入{\A section}相關的函數,以下列出特殊頁面或功能所需的程式碼
\begin{table}[H]\caption{首頁呈現}\label{wide}
    \centering
\begin{tabular}{cp{8cm}}
\extrarowheight=10pt
\\\hline
程式 & 效果\\
\hline
  $\backslash$ frametitle$\{$A$\}$  & 將A視為該頁簡報之標題
\\\hline  
  $\backslash$ framesubtitle$\{$A$\}$ & 將A視為該頁簡報之副標題
\\\hline

  $\backslash$ titlepage & 建立首頁
\\\hline
  $\backslash$ tableofcontents  & 建立目錄
\\\hline
  $\backslash$date$\{$日期$\}$ & 顯示日期,若為製作當天,可使用$\backslash$today,若需指定一天,則可以自行輸入與呈現的格式與時間,也可以使用$\backslash$empty讓日期不顯示  \\\hline
      \end{tabular}
    \end{table}
另外,若輸入數學符號、表格或圖片時,其指令皆一致,可參考前面章節
\section*{小結}
\LaTeX 簡報的製作論花俏的話,當然無法媲美Microsoft系列的Power Point,但是他表現的方式簡單明瞭,當然在數學函數的輸入上一樣無可匹敵,因此仍有一群喜好者,而且,在大型的數學或統計研討會上,真的都是\LaTeX 的天下,而且若結合R Markdowm,在R中跑的程式與結果便可以直接秀在簡報中,這樣方便的工具,真的還是很重要的。

\chapter{製作成書}
 將不同章節的\LaTeX 文件編排好後,可以使用\LaTeX 內部書目編排的指令,讓章節自動編排成書,也可以自動生成目錄、參照與索引等等,甚至還可以自動幫你編排成章節頁在左邊等等的功能,下列程式說明與範例參考自臺北大學汪群超老師的教學網頁\footnote{汪群超老師\LaTeX 教學官方網站 https://ntpuccw.blog/supplements/xetex-tutorial},以及維基百科\footnote{維基百科\LaTeX 教學 https://reurl.cc/pjg7e}網路資源\footnote{網路資源\LaTeX 教學 https://reurl.cc/VvEKn}。
\section{型態設定}
\LaTeX 製作書籍並不需要額外下載套件,僅須在preamble中設定$\backslash$documentclass$\{$book$\}$即可
\section{頁眉頁足}
書本的頁眉與頁足是畫龍點睛的,\LaTeX 中可以使用{\A fancyhdr}套件,便可以使用讓頁眉與頁足的表現更加多樣,可以在使用$\backslash$pagestyle$\{$fancy$\}$後,加入$\backslash$fancyhead$[$位置$]$ $\{$內容$\}$修改頁眉或$\backslash$fancyfoot$[$位置$]$ $\{$內容$\}$修改頁足,位置與內容可參考下表\\
\begin{table}[h]\caption{頁眉頁足位置}\label{fancy_site} 
    \centering
    \extrarowheight=8pt

        %加入標題與標號參照的文字
      %加高行高2pt
{\begin{tabular}{lc}
\extrarowheight=10pt 
\\\hline
程式& 表現   \\\hline
  E  & 偶數頁   \\\hline  % &代表換欄 \\代表換下一列
  O  & 基數頁   \\\hline
  L  & 左邊     \\\hline
  R  & 右邊     \\\hline
  C  & 中間     \\\hline
  H  & 上方     \\\hline
  F  & 下方     \\\hline
    \end{tabular}}       
\end{table}  
\begin{table}[h]\caption{頁眉頁足內容}\label{fancy_in} 
    \centering

{\begin{tabular}{lc}
\extrarowheight=10pt  
\\\hline
  $\backslash$thepage  & 此頁頁碼   \\\hline  % &代表換欄 \\代表換下一列
  $\backslash$leftmark  & 此chapter標題   \\\hline
  $\backslash$rightmark  & 小節標題     \\\hline
  $\backslash$thechapter  & 此chapter編號     \\\hline
  $\backslash$thesection  & 此section編號     \\\hline
    \end{tabular}}       
\end{table}  
    
\section{頁碼}
頁碼的使用在\LaTeX 中就很明顯的比使用WORD來的方便,不管是不要頁碼,還是要將頁碼編排的樣式更改,都簡單的指令便可以達到,使用$\backslash$pagenumbering$\{$編碼方式$\}$,編碼方式可以詳見下列表格,同時,也可以調整編碼起始的數值,僅需使用$\backslash$setcounter$\{$page$\}$ $\{$起始數值$\}$即可
\begin{table}[H]\caption{編碼方式}\label{numbering}
    \centering
{\begin{tabular}{lc}
\extrarowheight=10pt
\\\hline
程式 & 效果\\
\hline
  roman   & 羅馬字母
\\\hline  % &代表換欄 \\代表換下一列
  alph & 英文字母
\\hline
  arabic& 阿拉伯數字
\\\hline
  $\backslash$URCornerWallPaper & 右上角
\\\hline
  $\backslash$LRCornerWallPaper & 右下角
\\\hline
    \end{tabular}}
    \end{table}
    
\section{浮水印}
浮水印的製作可以讓頁面多的一些風味,使用{\A wallpaper}套件,並輸入$\backslash$位置$\{$圖片大小比例$\}$ $\{$圖片路徑與名稱$\}$,其中,位置的選擇可以有以下幾種,而在下列程式碼前加{\A This}則可以僅限定該頁的浮水印
種
\begin{table}[H]\caption{浮水印位置}\label{wallpaper_site}
    \centering
    
\begin{tabular}{lc}
\extrarowheight=10pt
\\\hline
程式 & 效果\\
\hline
  $\backslash$CenterWallPaper   & 頁面正中間
\\\hline  % &代表換欄 \\代表換下一列
  $\backslash$ULCornerWallPaper & 左上角
\\\hline
  $\backslash$ LLCornerWallPaper& 左下角
\\\hline
  $\backslash$URCornerWallPaper & 右上角
\\\hline
  $\backslash$LRCornerWallPaper & 右下角
\\\hline
    \end{tabular}
    \end{table}

\section{書內容}
書籍的內容編寫方式簡單,僅需依照上面幾張的方式便可以輸入漂亮的方程式、表格與圖片等等,然而,若已經將內容各自編排置其所屬的\LaTeX 文件檔中,可以使用$\backslash$include$\{$檔案名稱$\}$,不過此檔案需與書本編排的檔案存在同一資料夾中

\section{目錄}
在製作目錄的過程中,如果在書寫文件時有依照section等編排順序進行,那僅需在要出現目錄的地方輸入$\backslash$setcounter$\{$tocdepth$\}$ $\{$標題階層$\}$,在標題階層中輸入的數字可以決定哪些標題會出現在目錄中,接著,使用$\backslash$tableofcontents$($章節目錄$)$、$\backslash$listoffigures$($圖目錄$)$與$\backslash$listoftables$($表目錄$)$即可。

\section{索引}
索引和\index{目錄}目錄一樣,都能讓使用者在搜尋上更加方便,{\A makeidx}套件在\LaTeX 中便是用來製作索引的工具,必須在文件$\backslash$begin$\{$document$\}$前加入$\backslash$makeindex,在要呈現索引的地方輸入$\backslash$printondex,在要做為索引的文字前加入$\backslash$index$\{$該文字$\}$即可

\section*{小結}
書籍的編排還是建立在最一開始的數學函數、圖片等等使用,額外加的東西沒有很多,也算是一種整合型的功能,不過索引方面可能就不是很親民了,在編排上的確會遇到一些阻礙,還有進步的空間。
%\end{document}
